\documentclass[12pt]{article}
\usepackage{amsmath}
\usepackage[letterpaper,margin=1in]{geometry}
\usepackage{lipsum}
\pagenumbering{gobble}

% Tips for Applicants
% ===================
% * Indicate whether you are presenting a poster or oral presentation at the conference. If not, what presentations are you attending that benefit your academic and professional development? 
% * Are you sitting on a panel or serving another role at the conference (facilitator, etc.)?
% * How does this fit into your academic program? What are your post-graduation goals? How does this conference assist you in meeting those goals? 
% * How will you professionally benefit from participating in the program? How will your research outcomes positively impact the Biodesign Institute? 
% * Indicate how you plan on spending the grant (hotel, transportation, registration). 



% Applications must include: 
% ==========================
% * Personal statement detailing how this professional conference will benefit your academic and professional development; 
% * Your Individual Development Plan (IDP); and
% * Resume/CV.

\begin{document}

On Thursday, November 8 I will be travelling to the Australian National Univeristy (ANU) in Canberra, Australia for the ANU Computational Biology conference, where I will be giving an oral presentation on my work detecting somatic mutations in \textit{Eucalyptus} species. 
The conference is being organized by one of my close collaborators on this work, Dr. Robert Lanfear, who has invited me to present and is paying for travel to Canberra and lodging while at the conference. However, I will be responsible for meals while at the conference. I am applying for this award to pay for meals during the 6 days I will be in Australia. As the current per-diem for Canberra is \$126, I would spend the entirety of the \$500 award.

This conference will greatly assist me in meeting my academic goals. I am a PhD student in the School of Life Sciences Molecular and Cellular Biology (MCB) program. The program emphasizes learning to give oral presentations in conference and interview settings, and we practice this at our weekly colloquium held in the Biodesign auditorium. Attending this conference and presenting my work will allow me to continue developing my speaking skills and give me practice presenting in a more organic setting. It will also give me an opportunity to speak to an audience that is unfamiliar with my work. I can then get feedback on my work from a new audience. In particular, there will be many groups at the conference that are interested in the contribution of somatic mutations to plant evolution, so I will be able to discuss and present my work to other experts who are interested in similar questions.

As many of the people at the conference work in the same field and have the same research interests as I do, meeting them will greatly aid my professional goals. My current post-graduation goal is get a job as a reseach staff scientist or in industry. Talking and meeting with people who have experience in related fields will help me build my professional network and find potential employment opportunities. Meeting more people who themselves need qualified employees or know someone who does will greatly aid my ability to find a job after graduation.

Attending the conference will also benefit the Biodesign Institute. The Biodesign Institute emphasizes collaboration as one of the most important aspects of doing science, and giving this award will show to my collaborators and to everyone at the meeting that they are serious about their commitment to collaboration. That will lead other researchers at the meeting to desire more collaboration with Biodesign investigators, leading to more collaborations and more impactful science. On its own, my research will also positively impact Biodesign by leading to important discoveries about the role of somatic mutation in development, help uncover how somatic mutations contribute to cancer pathogenesis, and provide clues for how other organisms mitigate the negative affects of somatic mutation so that knowledge can be applied to advance human health.

\end{document}

